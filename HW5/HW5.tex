\documentclass{article}
\usepackage{graphicx} % Required for inserting images
\usepackage[utf8]{inputenc}
\usepackage{amsmath}
\usepackage{graphicx}
\usepackage{tikz}
\usepackage{array}
\usepackage{amssymb}
\newcommand*{\twoheadrightarrowtail}{\mathrel{\rightarrowtail\kern-1.9ex\twoheadrightarrow}}
% Alternative which doesn't look as good using the normal size, but might work better with smaller sizes too:
%\newcommand*{\twoheadrightarrowtail}{\mathrel{\rlap{$\rightarrowtail$}\twoheadrightarrow}}
\usepackage{amssymb}
\usepackage{amsthm}
\usepackage{multirow}
\usepackage{verbatim}
\usepackage{dcolumn}
\newcolumntype{2}{D{.}{}{2.0}}

\title{MATH 265 HW5}
\author{Hanzhang Yin}
\date{Sep/26/2024}

\begin{document}

\maketitle

\section*{Question 1}

\subsection*{(a)}
We can analysis this question by the parity of $n \in \mathbb{Z}_{+}$.
\begin{enumerate}
    \item When $n$ is odd, \((-1)^n = -1 \). Hence the term becomes:
    \\
    \[ - \left( 1 - \frac{1}{n} \right) = -1 + \frac{1}{n} \]
    When $n \rightarrow \infty$:
    \[ \frac{1}{n} \rightarrow 0 \Rightarrow -1 + \frac{1}{n} \rightarrow -1 \]
    Hence, for all odd $n$, \( -1 + \frac{1}{n} < -1 \)
    \item When $n$ is even, \((-1)^n = 1 \). Hence the term becomes:
    \[ \left( 1 - \frac{1}{n} \right) = 1 - \frac{1}{n} \]
    When $n \rightarrow \infty$:
    \[ \frac{1}{n} \rightarrow 0 \Rightarrow 1 - \frac{1}{n} \rightarrow 1 \]
    Hence, for all odd $n$, \( 1 - \frac{1}{n} \leq 1 \)
\end{enumerate}
Since every element $s \in S$, $s \leq 1$. By definition, $1$ is an upper bound of $S$.

\subsection*{(b)}

\begin{proof}
    If $M$ is an upper bound for $S$, then $M \geq s$, for every element $s \in S$.
    By our analysis in (a), it is sufficient to only consider the case when $n$ is even.
    \\
    Suppose $M < 1$ and $M$ is an upper bound of $S$.
    \[ n \rightarrow \infty \Rightarrow 1 - \frac{1}{n} \rightarrow 1 \]. 
    In this case, there exists a $s \in S$ s.t. $1 - \frac{1}{n} > M$ for some $n \in \mathbb{Z}_{+}$.
    This derived a contradiction, hence $M \geq 1$. 
\end{proof}

\subsection*{(c)}
\begin{proof}
    From part (a), $1$ is am upper bound of $S$, and from part (b), no number less then $1$ can be an upper bound.
    Thus by definition, the supremum of $S$ is $1$ (i.e. $\sup S = 1$).
\end{proof}


\section*{Quesiton 2}
Noting that by definition \( A + B = \{ a + b | a \in A, b \in B \} \)

\begin{proof}
    First we need to show that \( \sup(A + B) \leq \sup A + \sup B \):
    \\
    let $a \in A$ and $b \in B$. By definition of supremum, we have $a \leq \sup A$ and $b \leq \sup B$.
    \\
    Therefore, for any $a \in A$ and $b \in B$:
    \[ a + b \leq \sup A + \sup B \]
    Since \( A + B = \{ a + b | a \in A, b \in B \} \), every element in $A + B$ is less than or equal to $\sup A + \sup B$.
    Therefore, \( \sup \sup(A + B) \leq \sup A + \sup B \)
    \\
    Then we need to show \( \sup(A + B) \geq \sup A + \sup B \):
    \\
    Let $\epsilon > 0$, by definition of supremum, for set $A$ and $B$:
    \begin{itemize}
        \item \( \exists a_{\epsilon} \in A : \sup A - \epsilon < a_{\epsilon} < \sup A \)
        \item \( \exists b_{\epsilon} \in B : \sup B - \epsilon < b_{\epsilon} < \sup B \)
    \end{itemize}
    Now consider $a_{\epsilon} + b_{\epsilon} \in A + B$. Then:
    \[ (\sup A - \epsilon) + (\sup B - \epsilon) < a_{\epsilon} + b_{\epsilon} \leq \sup A + \sup B \]
    \[ \Rightarrow \sup A + \sup B - 2 \epsilon < a_{\epsilon} + b_{\epsilon} \leq \sup A + \sup B \]
    Noting that $\epsilon$ is arbitrary, as $\epsilon \rightarrow 0$:
    \[ \Rightarrow \sup A + \sup B < a_{\epsilon} + b_{\epsilon} \leq \sup A + \sup B \]
    Indicating $a_{\epsilon} + b_{\epsilon} \rightarrow \sup A + \sup B$
    Thus, $\sup A + \sup B$ is the supremum of $A + B$ by squeeze theorem. Therefore,
    \[ \sup A + \sup B \leq \sup (A + B) \]
    By combining two steps, we have $\sup A + \sup B = \sup (A + B)$
\end{proof}

\subsection*{Question 3}

\begin{proof}
    Let $x > 1$ and consider the set $S = \{ x^n : n \in \mathbb{Z}_{+} \}$.
    \\
    For any $M \in \mathbb{R}$, we want to show that there exists $n \in \mathbb{Z}_{+}$ such that $x^n > M$
    \\
    Taking natural logarithm on both sides of the inequality $x^n > M$, we get:
    \[ n \cdot \ln(x) > \ln(M) \]
    Since $\ln(x) > 0$, noting $x > 1$, we can solve for $n$:
    \[ n > \frac{\ln(M)}{\ln(x)} \]
    Note that RHS is a constant, let $N = \lceil \frac{\ln(M)}{\ln(x)} \rceil + 1$,
    \\
    Hence, for $n = N$, we have:
    \[ x^n > M \]
    For any $M \in \mathbb{R}$, we can found $n \in \mathbb{Z}_{+}$ s.t. $x^n > M$. Thus, the set is not bounded from above.
\end{proof}

\subsection*{Question 4}
For example:
\[ I_n = \left[ 1 + \frac{1}{n}, 4 - \frac{1}{n} \right) \]
For each positive integer $n$.
\\
Property verificaiton:
\begin{proof}
    \\
    \textbf{Part 1:} We show that \(\displaystyle \bigcup_{n=1}^{\infty} I_n = (1, 4)\).
    \\
    \textit{(i) \(\bigcup_{n=1}^{\infty} I_n \subseteq (1, 4)\)}:  
    Let \( x \in \bigcup_{n=1}^{\infty} I_n \). Then \( x \in I_n \) for some \( n \), i.e., \( 1 + \frac{1}{n} \leq x < 4 - \frac{1}{n} \). Since \( \frac{1}{n} > 0 \), it follows that \( 1 < x < 4 \). Thus, \( x \in (1, 4) \), and hence \(\bigcup_{n=1}^{\infty} I_n \subseteq (1, 4)\).
    \\
    \textit{(ii) \((1, 4) \subseteq \bigcup_{n=1}^{\infty} I_n\)}:  
    Let \( x \in (1, 4) \). Define \( \epsilon = \min\{ x - 1, 4 - x \} > 0 \). Choose \( N \in \mathbb{N} \) such that \( \frac{1}{N} < \epsilon \). Then, for all \( n \geq N \), we have \( 1 + \frac{1}{n} < x < 4 - \frac{1}{n} \), so \( x \in I_n \). Thus, \( x \in \bigcup_{n=1}^{\infty} I_n \), and hence \((1, 4) \subseteq \bigcup_{n=1}^{\infty} I_n\).
    \\
    Combining (i) and (ii), we obtain \(\bigcup_{n=1}^{\infty} I_n = (1, 4)\).
    
    \medskip
    
    \textbf{Part 2:} We show that \(\displaystyle \bigcap_{n=1}^{\infty} I_n = [2, 3]\).
    \\
    \textit{(i) \([2, 3] \subseteq \bigcap_{n=1}^{\infty} I_n\)}:  
    Let \( x \in [2, 3] \). For any \( n \in \mathbb{N} \), \( 1 + \frac{1}{n} \leq 2 \leq x \leq 3 \leq 4 - \frac{1}{n} \). Thus, \( x \in I_n \) for all \( n \), implying \( x \in \bigcap_{n=1}^{\infty} I_n \). Therefore, \([2, 3] \subseteq \bigcap_{n=1}^{\infty} I_n\).
    \\
    \textit{(ii) \(\bigcap_{n=1}^{\infty} I_n \subseteq [2, 3]\)}:  
    Let \( x \in \bigcap_{n=1}^{\infty} I_n \). Then \( x \in I_n \) for all \( n \), i.e., \( 1 + \frac{1}{n} \leq x < 4 - \frac{1}{n} \) for all \( n \). As \( n \to \infty \), \( 1 + \frac{1}{n} \to 1 \) and \( 4 - \frac{1}{n} \to 4 \). Thus, \( x \geq 2 \) (since \( x \) cannot be less than 2 for all large \( n \)) and \( x \leq 3 \) (since \( x \) cannot be greater than 3 for all large \( n \)). Hence, \( x \in [2, 3] \).
    \\
    Combining (i) and (ii), we obtain \(\bigcap_{n=1}^{\infty} I_n = [2, 3]\).
    
    \end{proof}

\end{document}
