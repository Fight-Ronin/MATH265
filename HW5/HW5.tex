\documentclass{article}
\usepackage{graphicx} % Required for inserting images
\usepackage[utf8]{inputenc}
\usepackage{amsmath}
\usepackage{graphicx}
\usepackage{tikz}
\usepackage{array}
\usepackage{amssymb}
\newcommand*{\twoheadrightarrowtail}{\mathrel{\rightarrowtail\kern-1.9ex\twoheadrightarrow}}
% Alternative which doesn't look as good using the normal size, but might work better with smaller sizes too:
%\newcommand*{\twoheadrightarrowtail}{\mathrel{\rlap{$\rightarrowtail$}\twoheadrightarrow}}
\usepackage{amssymb}
\usepackage{amsthm}
\usepackage{multirow}
\usepackage{verbatim}
\usepackage{dcolumn}
\newcolumntype{2}{D{.}{}{2.0}}

\title{MATH 265 HW5}
\author{Hanzhang Yin}
\date{Sep/26/2024}

\begin{document}

\maketitle

\section*{Question 1}

\subsection*{(a)}
We can analysis this question by the parity of $n \in \mathbb{Z}_{+}$.
\begin{enumerate}
    \item When $n$ is odd, \((-1)^n = -1 \). Hence the term becomes:
    \\
    \[ - \left( 1 - \frac{1}{n} \right) = -1 + \frac{1}{n} \]
    Noticing that for all $n$, $\frac{1}{n} \leq 1$, hence: 
    \[ -1 + \frac{1}{n} \leq 0 \]
    Hence, for all odd $n$, \( s \leq 0, s \in S \).
    \item When $n$ is even, \((-1)^n = 1 \). Hence the term becomes:
    \[ \left( 1 - \frac{1}{n} \right) = 1 - \frac{1}{n} \]
    When $n \rightarrow \infty$:
    \[ \frac{1}{n} \rightarrow 0 \Rightarrow 1 - \frac{1}{n} \rightarrow 1 \]
    Hence, for all odd $n$, \( 1 - \frac{1}{n} \leq 1 \)
\end{enumerate}
Since every element $s \in S$, $s \leq 1$. By definition, $1$ is an upper bound of $S$.

\subsection*{(b)}

\begin{proof}
    If $M$ is an upper bound for $S$, then $M \geq s$, for every element $s \in S$.
    \\
    Suppose $M < 1$ and $M$ is an upper bound of $S$, it is sufficient for us to only consider the case where $n$ is even.
    \\
    Since \( \frac{1}{2(1-M)} < 0 \), by Archimedes Property, \( \exists n \in \mathbb{N} \) s.t. \( \frac{1}{2(1-M)} < n \), we can get:
    \[ \frac{1}{2(1-M)} < n \Rightarrow \frac{1}{1 - M} < 2n \Rightarrow 1 < 2n(1 - M) \Rightarrow \frac{1}{2n} < 1 - M \Rightarrow M < 1 - \frac{1}{2n} \]
    In this case, we know that for $n$ is even,
    \[ M < 1 - \frac{1}{2n} = (1 - \frac{1}{2n})(-1)^{2n} \]
    Hence, there exists a $s \in S$ s.t. $M < (1 - \frac{1}{2n})(-1)^{2n}$ for some $n \in \mathbb{Z}_{+}$.
    This derived a contradiction, hence $M \geq 1$.
\end{proof}

\subsection*{(c)}
\begin{proof}
    From part (a), $1$ is am upper bound of $S$, and from part (b), no number less then $1$ can be an upper bound.
    Thus by definition, the supremum of $S$ is $1$ (i.e. $\sup S = 1$).
\end{proof}


\section*{Quesiton 2}
Noting that by definition \( A + B = \{ a + b \ | \ a \in A, b \in B \} \)

\begin{proof}
    First we need to show that \( \sup(A + B) \leq \sup A + \sup B \):
    \\
    let $a \in A$ and $b \in B$. By definition of supremum, we have $a \leq \sup A$ and $b \leq \sup B$.
    \\
    Therefore, for any $a \in A$ and $b \in B$:
    \[ a + b \leq \sup A + \sup B \]
    Since \( A + B = \{ a + b \ | \ a \in A, b \in B \} \), every element in $A + B$ is less than or equal to $\sup A + \sup B$.
    Therefore, \( \sup(A + B) \leq \sup A + \sup B \).
    \\
    Then we can derived the equality from here:
    \\
    Let $\epsilon > 0$, by definition of supremum, for set $A$ and $B$:
    \begin{itemize}
        \item \( \exists a_{\epsilon} \in A : \sup A - \epsilon < a_{\epsilon} < \sup A \)
        \item \( \exists b_{\epsilon} \in B : \sup B - \epsilon < b_{\epsilon} < \sup B \)
    \end{itemize}
    Now consider $a_{\epsilon} + b_{\epsilon} \in A + B$. Then:
    \[ (\sup A - \epsilon) + (\sup B - \epsilon) < a_{\epsilon} + b_{\epsilon} \leq \sup (A + B) \]
    \[ \Rightarrow \sup A + \sup B - 2 \epsilon < a_{\epsilon} + b_{\epsilon} \leq \sup (A + B) \]
    \[ \Rightarrow \sup A + \sup B - 2 \epsilon < \sup (A + B) \]
    \[ \Rightarrow 0 \leq \frac{1}{2}(\sup A + \sup B - \sup(A + B)) < \epsilon \]
    By previous step, WLOG, since \( \sup(A + B) \leq \sup A + \sup B \), the left inequality holds.
    Reffering to one of the theorem in section \textit{2.1}, we can get:
    \[ \frac{1}{2}(\sup A + \sup B - \sup(A + B)) = 0 \Rightarrow \sup A + \sup B = \sup(A + B) \]
\end{proof}

\subsection*{Question 3}

\begin{proof}
    Let $x > 1$ and consider the set $S = \{ x^n : n \in \mathbb{Z}_{+} \}$.
    \\
    For any $M \in \mathbb{R}$, we want to show that there exists $n \in \mathbb{Z}_{+}$ such that $x^n > M$
    \\
    Taking natural logarithm on both sides of the inequality $x^n > M$, we get:
    \[ n \cdot \ln(x) > \ln(M) \]
    Since $\ln(x) > 0$, noting $x > 1$, we can solve for $n$:
    \[ n > \frac{\ln(M)}{\ln(x)} \]
    Note that RHS is a constant, let $N = \lceil \frac{\ln(M)}{\ln(x)} \rceil + 1$,
    \\
    Hence, for $n = N$, we have:
    \[ x^n > M \]
    For any $M \in \mathbb{R}$, we can found $n \in \mathbb{Z}_{+}$ s.t. $x^n > M$. Thus, the set is not bounded from above.
\end{proof}

\subsection*{Question 4}
For example:
\[ I_n = \left[ 1 + \frac{1}{n}, 3 + \frac{1}{n} \right) \]
For each positive integer $n$.
\\
Property verificaiton:
\begin{proof}
\hspace*{0.001cm}
\\
\begin{enumerate}
    \item Prove that \( \bigcup_{n=1}^{\infty} I_n = (1, 4) \):
    \begin{itemize}
        \item Show that \( \bigcup_{n = 1}^{\infty} I_n \subseteq (1, 4) \):
        Let \( x \in \bigcup_{n = 1}^{\infty} I_n \). Then there exists $n \in \mathbb{N}$ s.t. $x \in I_n$, i.e.,
        \[ 1 + \frac{1}{n} \leq x < 3 + \frac{1}{n} \]
        Since \( \frac{1}{n} > 0 \), it follows that:
        \[ 1 < x < 3 + \frac{1}{n} \leq 3 + 1 = 4 \]
        Therefore, \( x \in (1, 4) \), and thus:
        \[ \bigcup_{n = 1}^{\infty} I_n \subseteq (1, 4) \]
        \item Show that \( (1, 4) \subseteq \bigcup_{n = 1}^{\infty} I_n \):
        Let \( x \in (1, 4) \). We need to find $n \in \mathbb{N}$ s.t. $x \in I_n$.
        \begin{itemize}
            \item If \( x \in (1, 3) \):
            Then \(x - 1 > 0 \). By Archimedes' Theorem, \( \exists N \in \mathbb{N} \) s.t.
            \[ \frac{1}{N} < x - 1 \]
            For \( n \geq N \), we have \( \frac{1}{n} \leq \frac{1}{N} < x - 1 \), so:
            \[ 1 + \frac{1}{n} < x \]
            Since \(x < 3\), and \( \frac{1}{n} > 0 \), we have:
            \[ x < 3 + \frac{1}{n} \]
            Therefore \( x \in I_n \), for all $n > N$.
            \item If \( x \in [3, 4) \):
            Then $4 - x > 0$. By Archimedes' Theorem, \( \exists N \in \mathbb{N} \) s.t.
            \[ \frac{1}{N} < 4 - x \]
            For all $n > N$ $\frac{1}{n} \leq \frac{1}{N} < 4 - x$, so:
            \[ x < 4 - \frac{1}{n} \]
            However, since the right endpoint of $I_n$ is $3 + \frac{1}{n}$, and $3 + \frac{1}{n} < 4 - \frac{1}{n}$,
            we need to ensure $x < 3 + \frac{1}{n}$.
            \\
            Observe that $x - 3 \geq 0$ and $x - 3 < 1$. By Archimedes theorem again, there exists $n \in \mathbb{N}$ s.t.
            \[ \frac{1}{n} > x - 3 \]
            This implies":
            \[ x - 3 < \frac{1}{n} \Rightarrow x < 3 + \frac{1}{n} \]
            Also, since $x \geq 3$ and $\frac{1}{n} > 0$:
            \[ x \geq 3 = 1 + 2 \leq 1 + \frac{1}{n} + 2, \ \text{since } \frac{1}{n} < 1 \]
            \textit{Note: $1 + \frac{1}{n} \leq x$ holds because $1 + \frac{1}{n} \leq 1 + 1 = 2 < x$}
            \\
            Therefore, $x \in I_n$
        \end{itemize}
    \end{itemize}
    Combining two steps, \( \bigcup_{n=1}^{\infty} I_n = (1, 4) \)

    \item Prove that \( \bigcap_{n=1}^{\infty} I_n = [2, 3] \):
    \begin{itemize}
        \item Show that $[2, 3] \subseteq \bigcap_{n=1}^{\infty} I_n$
        Let $x \in [2, 3]$. For all $n \in \mathbb{N}$, since:
        \( 1 + \frac{1}{n} \leq 1 + 1 = 2 \leq x \leq 3 \), and \( x < 3 + \frac{1}{n} \)
        we have:
        \[ 1 + \frac{1}{n} \leq x < 3 + \frac{1}{n} \]
        Therefore. $x \in I_n$, for all $n$. Thus,
        \[ [2, 3] \subseteq \bigcap_{n=1}^{\infty} I_n \]
        \item Show that $ \bigcap_{n=1}^{\infty} I_n \subseteq [2, 3] $
        Let $x \in \bigcap_{n=1}^{\infty} I_n$. Then for all $n \in \mathbb{N}$:
        \[ 1 + \frac{1}{n} \leq x < 3 + \frac{1}{n} \]
        \begin{itemize}
            \item Show $x \geq 2$:
            Suppose, $x < 2$, then $x - 1 < 1$. By Archimedes Theorem, there exists $N \in \mathbb{N}$ s.t. $\frac{1}{N} < x - 1$.
            \\
            For all $n \geq N$:
            \[ 1 + \frac{1}{n} < x \]
            which contradicts the fact that $x \geq 1 + \frac{1}{n}$, for all $n$. Therefore, $x \geq 2$.
            \item Show $x \leq 3$:
            Suppose, $x > 3$, then $x - 3 > 0$. By Archimedes Theorem, there exists $N \in \mathbb{N}$ s.t. $\frac{1}{N} < x - 3$.
            \\
            For all $n \geq N$:
            \[ x > 3 + \frac{1}{n} \]
            which contradicts the fact that $x < 1 + \frac{1}{n}$, for all $n$. Therefore, $x \leq 3$.
        \end{itemize}
        Thus, $x \in [2, 3]$, so:
        \[ \bigcap_{n=1}^{\infty} I_n \subseteq [2, 3] \]
    \end{itemize}
    Combining both inclusions, we have:
    \[ \bigcap_{n=1}^{\infty} I_n = [2, 3] \]
\end{enumerate}
Overall \( I_n = \left[ 1 + \frac{1}{n}, 3 + \frac{1}{n} \right) \) satisfy two given properties.

\end{proof}

\end{document}
