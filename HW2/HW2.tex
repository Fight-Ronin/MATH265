\documentclass{article}
\usepackage{graphicx} % Required for inserting images
\usepackage[utf8]{inputenc}
\usepackage{amsmath}
\usepackage{graphicx}
\usepackage{tikz}
\usepackage{array}
\usepackage{amssymb}
\newcommand*{\twoheadrightarrowtail}{\mathrel{\rightarrowtail\kern-1.9ex\twoheadrightarrow}}
% Alternative which doesn't look as good using the normal size, but might work better with smaller sizes too:
%\newcommand*{\twoheadrightarrowtail}{\mathrel{\rlap{$\rightarrowtail$}\twoheadrightarrow}}
\usepackage{amssymb}
\usepackage{amsthm}
\usepackage{multirow}
\usepackage{dcolumn}
\newcolumntype{2}{D{.}{}{2.0}}

\title{MATH 265 HW2}
\author{Hanzhang Yin}
\date{Sep/4/2024}

\begin{document}

\maketitle

\section*{Question 1}
\begin{proof}
    First lets check the base case. For $n = 1$: 
    \[ \frac{1}{\sqrt{1}} = 1 > \frac{1^{\frac{3}{2}}}{3} = \frac{1}{3} \]
    Thus, the base case holds.
    \\
    For forming up the inductive hypothesis, assume the statement is true for somme $n = k$, 
    \[ \frac{1}{\sqrt{1}} + \frac{1}{\sqrt{2}} + \cdots + \frac{1}{\sqrt{k}} \geq \frac{k^{\frac{3}{2}}}{3} \]
    \\
    Now for the inductive step, we need to show the statement holds for $n = k + 1$,
    \[  \frac{1}{\sqrt{1}} + \frac{1}{\sqrt{2}} + \cdots + \frac{1}{\sqrt{k}} + \frac{1}{\sqrt{k + 1}} \geq \frac{(k+1)^{\frac{3}{2}}}{3} \]
    \\
    From inductive hypothesis, we can add $\frac{1}{\sqrt{k+1}}$ both side:
    \[ \frac{1}{\sqrt{1}} + \frac{1}{\sqrt{2}} + \cdots + \frac{1}{\sqrt{k}} + \frac{1}{\sqrt{k + 1}} \geq \frac{(k)^{\frac{3}{2}}}{3} + \frac{1}{\sqrt{k + 1}} \]
    \[ \]
\end{proof}


\end{document}