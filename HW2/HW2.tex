\documentclass{article}
\usepackage{graphicx} % Required for inserting images
\usepackage[utf8]{inputenc}
\usepackage{amsmath}
\usepackage{graphicx}
\usepackage{tikz}
\usepackage{array}
\usepackage{amssymb}
\newcommand*{\twoheadrightarrowtail}{\mathrel{\rightarrowtail\kern-1.9ex\twoheadrightarrow}}
% Alternative which doesn't look as good using the normal size, but might work better with smaller sizes too:
%\newcommand*{\twoheadrightarrowtail}{\mathrel{\rlap{$\rightarrowtail$}\twoheadrightarrow}}
\usepackage{amssymb}
\usepackage{amsthm}
\usepackage{multirow}
\usepackage{dcolumn}
\newcolumntype{2}{D{.}{}{2.0}}

\title{MATH 265 HW2}
\author{Hanzhang Yin}
\date{Sep/4/2024}

\begin{document}

\maketitle

\section*{Question 1}
\begin{proof}
    First lets check the base case. For $n = 1$: 
    \[ \sum_{i=1}^{1} \frac{1}{\sqrt{i}} = \frac{1}{\sqrt{1}} = 1 \]
    $1^{\frac{2}{3}} = 1$. Since $1 \geq 1$, the base case holds.
    \\
    For forming up the inductive hypothesis, assume the statement is true for some $k \geq 1$, 
    \[ \sum_{i=1}^{k} \frac{1}{\sqrt{i}} \geq k^{\frac{2}{3}}  \]
    \\
    Now we need to show the statement holds for $k + 1$, namely, 
    \[ \sum_{i=1}^{k+1} \frac{1}{\sqrt{i}} \geq (k + 1)^{\frac{2}{3}} \]
    \\
    From inductive hypothesis, we can add $\frac{1}{\sqrt{k+1}}$ both side:
    \[ \sum_{i=1}^{k+1} \frac{1}{\sqrt{i}} + \frac{1}{\sqrt{k + 1}} \geq (k + 1)^{\frac{2}{3}} + \frac{1}{\sqrt{k + 1}} \]
    \\
    Now we need to show:
    \[ k^{\frac{2}{3}} + \frac{1}{\sqrt{k + 1}} \geq (k + 1)^{\frac{2}{3}} \]
    \\
    Using approximation of power functions (Taylor Expansion), we can approximate RHS: 
    \[ (k + 1)^{\frac{2}{3}} \approx k^{\frac{2}{3}} + \frac{2}{3k^{\frac{1}{3}}} \]
    \\
    When $k$ is large, we can approximate LHS:
    \[ k^{\frac{2}{3}} + \frac{1}{\sqrt{k + 1}} \approx k^{\frac{2}{3}} + \frac{1}{k^{\frac{1}{2}}} \]
    \\
    For, \( \frac{1}{k^{\frac{1}{2}}} \leq \frac{2}{3k^{\frac{1}{3}}} \) since the former diminishes slower. Hence,
    \[ k^{\frac{2}{3}} + \frac{1}{k^{\frac{1}{2}}} \geq k^{\frac{2}{3}} + \frac{2}{3 k^{\frac{2}{3}}} \Rightarrow k^{\frac{2}{3}} + \frac{1}{\sqrt{k + 1}} \geq (k + 1)^{\frac{2}{3}} \]
    \\
    By mathematical induction, the statement \( \sum_{i=1}^{k} \frac{1}{\sqrt{i}} \geq n^{\frac{2}{3}} \) is true for all $n in \mathbb{N}$.
\end{proof}

\section*{Question 2}
\begin{proof}
    First lets check the base case. For $n = 0$: 
    \[ x_0 = 3, x_1 = \frac{1}{8} \cdot (3)^2 + 2 = \frac{9}{8} + 2 = \frac{25}{8} = 3.125 \]
    \\
    $x_0 < x_1 < 4$, the base case proved.
    \\
    Assume that for some $n = k, k \geq 1$, $x_k < x_{k+1} < 4$. WTS $x_{k+1} < x_{k+1=2} < 4$. 
    \\
    From the recurrence relation: \[ x_{k+2} = \frac{1}{8}x^2_{k+1} + 2 \]
    \\
    Using the inductive hypothesis, note that $x_{k+1} < 4$:
    \[ x_{k+2} = \frac{1}{8}x^2_{k+1} + 2 < \frac{1}{8}(4^2) + 2 = 2 + 2 = 4 \]
    \\
    By mathematical induction, the statement $x_{n} < x_{n + 1} < 4$ is true for all $n \in \mathbb{N} \cup \{0\}$.
\end{proof}

\section*{Question 3}
\begin{proof}
    First lets check the base case. For $k = 1$: 
    \[ F_{m + 1} = F_{m - 1}F_1 + F_m F_2 \]
    \\
    By definition of Fibonacci Sequence, $F_1 = F_2 = 1$.
    \[ F_{m + 1} = F_{m - 1} + F_m \]
    This is true by definition, hence the base case proved.
    \\
    Assume that for some $k \geq 1$, the given identity holds:
    \[ F_{m + k} = F_{m - 1}F_k + F_m F_{k + 1} \]
    \\
    We need to prove that this statement holds for $k+1$, namely,
    \[ F_{m + k + 1} = F_{m - 1}F_{k + 1} + F_m F_{k + 2} \]
    \\
    Using the Fibonacci sequence's definition, we express $F_{k+2}$ and $F_{m+k+1}$ as:
    \[ F_{k+2} = F_{k+1} + F_k \]
    \[ F_{m+k+1} = F_{m+k} + F_{m+k-1} \]
    \\
    Substitute the values from the inductive hypothesis into $F_{m+k+1}$:
    \[ F_{m+k+1} = (F_{m-1}F_k + F_mF_{k+1}) + (F_{m-1}F_{k-1} + F_mF_k) \]
    \\
    Combine and reorganize terms:
    \[  F_{m+k+1} = F_{m-1}(F_k + F_{k-1}) + F_m(F_{k+1} + F_k) \]
    \\
    By Finbonacci definition:
    \[ F_{m+k+1} = F_{m-1}F_{k+1} + F_m(F_{k+1} + F_k) \]
    \[ =  F_{m-1}F_{k+1} + F_mF_{k+2} \]
    \\
    By mathematical induction, the statement is true for all $k.m \in \mathbb{N}$ with $m \geq 2$.
\end{proof}


\section*{Question 4}
\begin{proof}
    Let $A_n$ := \{polynomials of degree $n$ over $\mathbb{Z}$\}
    \\
    Since $\mathbb{Z}$ is countably infinite and product of countably infinite set is countably infinite, we want to show the bijection between:
    \[ A_n \hookrightarrow \prod_{i=0}^{n}\mathbb{Z}_{i}: a_n x^n + \cdots + a_1 x + a_0 \mapsto (a_n, \cdots, a_1, a_0) \]
    
    \begin{itemize}
        \item Prove $A_n$ is \textit{one to one}
        \begin{proof}
            Let $p(x) = a_n x^n + \cdots + a_1 x + a_0$
            To prove $A_n$ is \textit{one to one}, we must show that if $A_n(p(x_1)) = A_n(p(x_2))$, then $p(x_1) = p(x_2)$.
            \\
            Suppose $A_n(p(x_1)) = A_n(p(x_2))$. Then:
            \[ (a_{n_1}, \cdots, a_{1_1}, a_{0_1}) = (a_{n_2}, \cdots, a_{1_2}, a_{0_2}) \]
            \\
            By definition of tuple equality,
            \[ a_{n_1} = a_{n_2}, \cdots, a_{0_1} = a_{0_2}\]
            \\
            Since the coefficient of $p(x_1)$ and $p(x_2)$ are identical, the function $A_n$ is injective.
        \end{proof}
        \item Prove $A_n$ is \textit{onto}
        \begin{proof}
            To prove $A_n$ is \textit{onto}, we must show that for every tuple $(a_n, \cdots, a_0) \in \prod_{i=0}^{n}\mathbb{Z}_{i}$, there exists a polynomial $p(x) \in A_n$ s.t. $f(p(x)) = (a_n, \cdots, a_0)$.
            \\
            For any given tuple $(a_n, \cdots, a_0)$, we can find a polynomial 
            \[ p(x) = a_n x^n + \cdots + a_1 x + a_0 \ s.t. \ f(p(x)) = (a_n, \cdots, a_0) \]
            \\
            Hence, $A_n$ is surjective by definition
        \end{proof}
    \end{itemize}
    Note that $\mathbb{Z}$ is countably infinite, there exists a bijection between $\mathbb{Z}$ and $\mathbb{N}$. We also know that $\mathbb{N} \times \mathbb{N}$ is countably infinite. Hence, we can get:
    \[ A_n \hookrightarrow \prod_{i=0}^{n}\mathbb{Z}_{i} \hookrightarrow \mathbb{N} \times \mathbb{N} \]
    \\
    Therefore, algebraic numbers is countably infinite.
\end{proof}

\end{document}