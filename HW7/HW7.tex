\documentclass{article}
\usepackage{graphicx} % Required for inserting images
\usepackage[utf8]{inputenc}
\usepackage{amsmath}
\usepackage{graphicx}
\usepackage{tikz}
\usepackage{array}
\usepackage{amssymb}
\newcommand*{\twoheadrightarrowtail}{\mathrel{\rightarrowtail\kern-1.9ex\twoheadrightarrow}}
% Alternative which doesn't look as good using the normal size, but might work better with smaller sizes too:
%\newcommand*{\twoheadrightarrowtail}{\mathrel{\rlap{$\rightarrowtail$}\twoheadrightarrow}}
\usepackage{amssymb}
\usepackage{amsthm}
\usepackage{multirow}
\usepackage{verbatim}
\usepackage{dcolumn}
\newcolumntype{2}{D{.}{}{2.0}}

\title{MATH 265 HW7}
\author{Hanzhang Yin}
\date{Oct/18/2024}

\begin{document}

\maketitle

\section*{Question 1}

\subsection*{(a)}
\begin{proof}
    Let $x > 1$ and $y \in \mathbb{Q}$. We aim to show that $x^y$ is the l.u.b of the set
    \[ E(x, y) = \{ x^t | t < y, t \in \mathbb{Q} \} \]
    \begin{enumerate}
        \item Upper Bound: 
        \\
        For any $t \in \mathbb{Q}$ with $t < y$, given $x > 1$, $x^t$ is increasing in $t$, we have:
        \[ x^t < x^y \]
        Therefore, $x^y$ is an upper bound of $E(x, y)$
        \item Least Upper Bound (Supremum): 
        \\
        Suppose there exists a real number $M < x^y$ that is also an upper bound of $E(x,y)$
        Since $x^y - M > 0$, set $\epsilon = x^y - M$.
        \\
        Noting that rationals are dense in $\mathbb{R}$, there exists $t \in \mathbb{Q}$ s.t. 
        \[ y - \delta < t < y \]
        Where $\delta > 0$ is small enough to ensure $x^y - x^t < \epsilon$.
        Then:
        \[ x^t > x^y - \epsilon = M \]
        This contradict the assumption that $M$ is an upper bound of $E(x, y)$.
        Therefore, no number less than $x^y$ can be an upper bound of $E(x,y)$, so $x^y$ is the supremum of $E(x,y)$.
    \end{enumerate}
\end{proof}

\subsection*{(b)}
\begin{proof}
    To show boundness, we need to show there exists an upperbound and lower bound in $E(x,y)$, when $y \in \mathbb{R}$
    \\
    \begin{enumerate}
        \item Upper bound: 
        \\
        Since $x^t < x^y$ for all $t < y$, $x^y$ is an upper bound of $E(x,y)$
        \item Lower Bound:
        \\
        And we know that $x^t > 0$ for all $t$ by given definition. Therefore, $E(x,y)$ is bounded below.
    \end{enumerate}
    Hence $E(x, y)$ is bounded.
\end{proof}

\subsection*{(c)}
\begin{proof}
    To proof $x^{y + z} = x^y x^z$, we need to validate two directions:
    \\
    \begin{enumerate}
        \item First we need to show $x^{y + z} \leq x^y x^z$:
        \\
        Let $t \in \mathbb{Q}$ with $t < y + z$.
        Since $\mathbb{Q}$ is dense in $\mathbb{R}$, there exists $t_1, t_2 \in \mathbb{Q}$ s.t.:
        \[ t_1 < y; \ t_2 < z; \ t = t_1 + t_2 \]
        Then:
        \[ x^t = x^{t_1 + t_2} = x^{t_1}x^{t_2} < x^yx^z \]
        Therefore, every element $x^t$ of $E(x, y + z) \leq x^yx^z$
        Thus:
        \[ x^{y + z} = \sup E(x, y + z) \leq x^y x^z \]
        \item Then we need to show $x^{y + z} \geq x^y x^z$:
        For any $\epsilon > 0$, choost $t_1, t_2 \in \mathbb{Q}$ s.t.:
        \[ y - \epsilon < t_1 < y, \ z - \epsilon < t_2 < z \]
        Then $t_1 + t_2 < y + z$, so $t_1 + t_2 \in E(x, y + z)$
        Compute: 
        \[ x^{t_1}x^{t_2} = x^{t_1 + t_2} \leq x^{y + z} \]
        Since $t_1 \rightarrow y$ and $t_2 \rightarrow z$, $x^{t_1} \rightarrow x^y$ and $x^{t_2} \rightarrow x^z$.
        \\
        Therefore:
        \[ x^{y + z} \geq \lim_{t_1 \rightarrow y^{-}} \lim_{t_2 \rightarrow z^{-}} x^{t_1} x^{t_2} = x^y x^z - \delta \]
        where $\delta$ can be made arbitrarily small. Thus:
        \[ x^{y + z} \geq x^y x^z \]
    \end{enumerate}
    Combining both inequalities, we get $x^{y + z} = x^y x^z$
    \\
    The property \( x^{y+z} = x^y x^z \) implies that the function \( f(y) = x^y \) is \textbf{injective} because it ensures that equal outputs correspond to equal inputs. 
    Specifically by definition, if \( f(y_1) = f(y_2) \), then \( x^{y_1} = x^{y_2} \). Using the exponential property, we have \( x^{y_1 - y_2} = 1 \). Since \( x > 1 \) and the exponential function \( x^t \) is strictly increasing, the equation \( x^t = 1 \) holds only when \( t = 0 \). Therefore, \( y_1 - y_2 = 0 \), which means \( y_1 = y_2 \). 
    This shows that \( f \) is injective because no two different inputs produce the same output.
\end{proof}

\section*{Question 2}
\begin{proof}
    \textit{The following proof follows the given proof outline:}
    \\
    \textbf{Step i:} For any $n \in \mathbb{N}$, $x - 1 \geq n(x^{\frac{1}{n}} - 1)$
    \\
    Recall Bernoulli's inequality, it states that for any real number $r \geq 1$ and $s \geq -1$:
    \[ (1 + s)^r \geq 1 + rs \]
    Let $s = x^{\frac{1}{n}} - 1$ (NOTE: $s > 0$ as $x > 1$) and $r = n$:
    \[ (1 + x^{\frac{1}{n}} - 1)^n \geq 1 + n(x^{\frac{1}{n}} - 1) \]
    \[ \Rightarrow x \geq 1 + n(x^{\frac{1}{n}} - 1) \Rightarrow x - 1 \geq n(x^{\frac{1}{n}} - 1) \]
    \textbf{Step ii:} If $t > 1$ and $n \in \mathbb{N}$ s.t. $n > \frac{x - 1}{t - 1}$, then $x^\frac{1}{n} < t$.
    \\
    From \textit{step i}, we get:
    \[ x^{\frac{1}{n}} - 1 \leq \frac{x - 1}{n} \]
    If $n > \frac{x - 1}{t - 1}$, then:
    \[ \frac{x - 1}{n} < t - 1 \]
    Therefore,
    \[ x^{\frac{1}{n}} - 1 < t - 1 \Longrightarrow x^{\frac{1}{n}} < t \]
    \textbf{Step iii:} If $y \in \mathbb{R}$ and $x^y < z$, then there exists $n \in \mathbb{N}$ s.t. $x^{y + \frac{1}{n}} < z$:
    \\
    Set $t = \frac{x^y}{z}$, note $\frac{x^y}{z} > 1$ by definition. Again using step ii, choose $n \in \mathbb{N}$ s.t.:
    \[ n > \frac{x - 1}{t - 1} \]
    Then $x^{\frac{1}{n}} < t$, recall question $1(c)$ and multiply $x^y$ on both side, we can get:
    \[ x^{y + \frac{1}{n}} = x^y \cdot x^{\frac{1}{n}} < x^y \cdot t = z \]
    \textbf{Step iv:} If $y \in \mathbb{R}$ and $x^y > z$, then there exists $n \in \mathbb{N}$ s.t. $x^{y + \frac{1}{n}} > z$:
    \\
    Set $t = \frac{z}{x^y}$, note $\frac{z}{x^y} > 1$ by definition. Again using step ii, choose $n \in \mathbb{N}$ s.t.:
    \[ n > \frac{x - 1}{t - 1} \]
    Then $x^{\frac{1}{n}} < t$, so $x^{-\frac{1}{n}} = \frac{1}{x^{\frac{1}{n}}} > \frac{1}{n}$. Recall question $1(c)$ and multiply $x^y$ on both side, we can get:
    \[ x^{y - \frac{1}{n}} = x^y \cdot x^{-\frac{1}{n}} < x^y \cdot \frac{1}{t} = z \]
    \textbf{Step v:} Define $A(z) = \{w \in \mathbb{R} | x^w < z \}$. Let $y = \sup A(z)$. then $x^y = z$
    \\
    First, since $x^w \rightarrow 0$, as $w \rightarrow - \infty$ and $x^w \rightarrow \infty$, as $w \rightarrow \infty$, there exist real numbers $w$ s.t. $x^w < z$.
    \\
    The set $A(z)$ is bounded above because $x^w \geq z$ for sufficiently large $w$.
    \\
    Suppose $x^y < z$: by step (iii), there exists $n$ s.t. $x^{y + \frac{1}{n}} < z$, contradicting the fact that $y = \sup A(z)$.
    \\
    Similarly, suppose $x^y > z$: by step (iv), there exists $n$ s.t. $x^{y - \frac{1}{n}} > z$, but $y - \frac{1}{n} < y$, again contradicting the fact that $y = \sup A(z)$ (i.e. $y$ is l.u.b).
    \\
    Consequently, neither $x^y < z$ not $x^y > x$ is possible; thus, $x^y = z$.
\end{proof}

\section*{Question 3}
By analyzed several terms of the given sequence, likely it is increasing and should converges to $2$.
\begin{proof}
    For the given sequence, we can define it recusively:
    \[ a_1 = \sqrt{2}, \ a_{n + 1} = \sqrt{2a_n}, \ for \ n \geq 1 \]
    Then we can set an induction to prove this sequence is increasing. I.e. We will prove $a_n < a_{n + 1}$ for all $n \geq 1$
    \\
    Base Case (n = 1):
    \\
    \[ a_1 = \sqrt{2}, \ a_2 = \sqrt{2a_1} = \sqrt{2\sqrt{2}} \]
    Noticing that $a_1, a_2 > 0$, and $a_1^2 = 2, a_2^2 = 2\sqrt{2}$, so:
    \[ 2 < 2\sqrt{2} \Leftrightarrow a_1^2 < a_2^2 \Rightarrow a_1 < a_2 \]
    \\
    Inductive steps:
    \\
    Assume $a_n < a_{n + 1}$ for some $n \geq 1$. We need to show $a_{n + 1} < a_{n + 2}$.
    \\
    Noticing, $a_{n + 1}$ and $a_{n + 2}$ can be deonte as:
    \[ a_{n + 1} = \sqrt{2a_n}, \ a_{n + 2} = \sqrt{2a_{n+1}} \]
    Then, since $a_n > 0, \ \forall n \in \mathbb{N}$:
    \[ \sqrt{2a_n} < \sqrt{2a_{n+1}} \Leftrightarrow 2a_n < 2a_{n+1} \Leftrightarrow a_n < a_{n+1} \]
    Hence, by our induction hypothesis, $a_{n + 1} < a_{n + 2}$. Therefore by induction, sequence $\{a_n\}$ is increasing.
    \\
    Then, we need to show that $(a_n)$ is bounded above. I.e., we will prove by induction that $a_n < 2$ for all $n > 1$
    \\
    Base Case (n = 1):
    \\
    \[ a_1 = \sqrt{2} < 2 \]
    \\
    Inductive Step:
    \\
    Assume $a_n < 2$ for some $n \geq 1$, we need to show $a_{n + 1} < 2$:
    \\
    From our previous definition, $a_{n + 1} = \sqrt{2 a_n}$. 
    Since $a_n < 2$, we have $2a_n < 2 \times 2 = 4$, hence:
    \[ a_{n + 1} = \sqrt{2a_n} < \sqrt{4} = 2 \]
    By induction again, $a_n < 2$ for all $n > 1$.
    \\
    Since $\{a_n \}$ is increasing and bounded above by $2$, by MCT, $\{a_n\}$ converges to some limit $L$ s.t. $L \leq 2$
    \\
    Let $L = \lim_{n \rightarrow \infty} a_n$. Taking limit both sides of the recursive formula:
    \[ L = \lim_{n \rightarrow \infty} a_{n + 1} = \lim_{n \rightarrow \infty} \sqrt{2a_n} = \sqrt{2L} \]
    Then:
    \[ L = \sqrt{2L} \Longrightarrow L^2 = 2L \Longrightarrow L^2 - 2L = 0 \Longrightarrow L(L - 2) = 0 \]
    So $L = 0, 2$, and since $a_n > 0$, limit $L \geq \sqrt{2} > 0$. so $L = 2$.
    \\
    Lastly, the limit found for the given sequence is:
    \[ \lim_{n \rightarrow \infty} a_n = 2 \]    
\end{proof}

\end{document}
