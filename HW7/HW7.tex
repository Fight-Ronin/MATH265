\documentclass{article}
\usepackage{graphicx} % Required for inserting images
\usepackage[utf8]{inputenc}
\usepackage{amsmath}
\usepackage{graphicx}
\usepackage{tikz}
\usepackage{array}
\usepackage{amssymb}
\newcommand*{\twoheadrightarrowtail}{\mathrel{\rightarrowtail\kern-1.9ex\twoheadrightarrow}}
% Alternative which doesn't look as good using the normal size, but might work better with smaller sizes too:
%\newcommand*{\twoheadrightarrowtail}{\mathrel{\rlap{$\rightarrowtail$}\twoheadrightarrow}}
\usepackage{amssymb}
\usepackage{amsthm}
\usepackage{multirow}
\usepackage{verbatim}
\usepackage{dcolumn}
\newcolumntype{2}{D{.}{}{2.0}}

\title{MATH 265 HW7}
\author{Hanzhang Yin}
\date{Oct/18/2024}

\begin{document}

\maketitle

\section*{Question 1}

\subsection*{(a)}
\begin{proof}
    Let $x > 1$ and $y \in \mathbb{Q}$. We aim to show that $x^y$ is the l.u.b of the set
    \[ E(x, y) = \{ x^t | t < y, t \in \mathbb{Q} \} \]
    \begin{enumerate}
        \item Upper Bound: 
        \\
        For any $t \in \mathbb{Q}$ with $t < y$, given $x > 1$, $x^t$ is increasing in $t$, we have:
        \[ x^t < x^y \]
        Therefore, $x^y$ is an upper bound of $E(x, y)$
        \item Least Upper Bound (Supremum): 
        \\
        By the consequence of the Density Theorem: For $M \in \mathbb{R}$ satisfying $M < x^y$, there exists $t \in \mathbb{Q}$ such that $M < x^t < x^y$.
        \\
        Suppose $M$ is an upper bound of set $E(x,y)$ and $M < x^y$, by the consequence listed upwards, we can always find an $x^t$ s.t. $x^t > M$. This contradict the assumption that
        $M$ is an upper bound of $E(x, y)$. Therefore, no number less than $x^y$ can be an upperbound of $E(x,y)$. Hence, $x^ = \sup E(x, y)$.
    \end{enumerate}
\end{proof}

\subsection*{(b)}
\begin{proof}
    Given $y \in \mathbb{R}$, since $\mathbb{Q}$ is dense in $\mathbb{R}$, $\exists y' \in \mathbb{Q}$ s.t. $y < y'$.
    Hence,
    \[ E(x, y) \subseteq E(x, y') \]
    Which is equivilant saying by definition, for all $t < y$,
    \[ 0 < x^t < x^y' \]
    Thus, $E(x, y)$ is bounded for $y \in \mathbb{R}$.
\end{proof}

\subsection*{(c)}
\begin{proof}
    First, we need to show that let $t \in \mathbb{Q}$ with $t < y + z$, there exists $t_1, t_2 \in \mathbb{Q}$ s.t. $t = t_1 + t_2, \ t_1 < y, \ t_2 < z$.
    \\
    By densitiy theorem, $\exists t_1 \in \mathbb{Q}$ s.t.
    \[ y - \epsilon < t_1 < y \ (1) \] 
    we pick $\epsilon = y + z - t$ for the following proof.
    let $t_2 = t - t_1$ (NOTE: Since $t, t_1 \in \mathbb{Q}$, then $t_2 \in \mathbb{Q}$). 
    \\
    Then we sub $t_1 = t - t_2$ and our defined $\epsilon$ into (1), we can get:
    \[ y - \epsilon < t - t_2 < y \Rightarrow y - y - z + t < t - t_2 < y \Rightarrow t - z < t - t_2 < y \]
    \[ \Rightarrow -z < -t_2 < y - t \Rightarrow t_2 < z < y - t \]
    By looking at the left inequality, we finished our proof.
    \\
    Using our proved statement upwards, to proof $x^{y + z} = x^y x^z$, we need to validate two directions:
    \\
    \begin{enumerate}
        \item First we need to show $x^{y + z} \leq x^y x^z$:
        \\
        Let $t \in \mathbb{Q}$ with $t < y + z$.
        Since $\mathbb{Q}$ is dense in $\mathbb{R}$, there exists $t_1, t_2 \in \mathbb{Q}$ s.t.:
        \[ t_1 < y; \ t_2 < z; \ t = t_1 + t_2 \]
        Then:
        \[ x^t = x^{t_1 + t_2} = x^{t_1}x^{t_2} < x^yx^z \]
        Therefore, every element $x^t$ of $E(x, y + z) \leq x^yx^z$.
        Thus:
        \[ x^{y + z} = \sup E(x, y + z) \leq x^y x^z \]
        \item Then we need to show $x^{y + z} \geq x^y x^z$:
        For any $t < y, u < z$, $t,u \in \mathbb{Q}$, by 1 (c), we can get:
        \[ x^{t + u} \leq x^{y + z} \Leftrightarrow x^t x^u \leq x^{y + z} \]
        (NOTE: $x^{t + u} \in E(x, y + z)$)
        \\
        We can rearrange this inequality into 2 forms:
        \begin{enumerate}
            \item \[ x^u \leq x^{-t} x^{y + z} \Rightarrow x^z \leq x^{-t} x^{y + z} \]
            Since $x^z$ is l.u.b by definition and $x^{-t} x^{y + z}$ is an upperbound.
            \item \[ x^t \leq x^{-u} x^{y + z} \Rightarrow x^z \leq x^{-u} x^{y + z} \]
            Similarly for this case.
        \end{enumerate}
        Therefore, combining two inequalities,
        \[ x^z x^y \leq (x^{-t})(x^{-u})x^{y + z} \leq x^{y + z} \Rightarrow x^{y + z} \leq x^z x^y \]
    \end{enumerate}
    Combining both inequalities, we get $x^{y + z} = x^y x^z$
    \\
    The property \( x^{y+z} = x^y x^z \) implies that the function \( f(y) = x^y \) is \textbf{injective}.
    Specifically by definition, if \( f(y_1) = f(y_2) \), then \( x^{y_1} = x^{y_2} \). Using the exponential property proved in (c), we have \( x^{y_1 - y_2} = 1 \). Since \( x > 1 \) and the exponential function \( x^t \) is strictly increasing, the equation \( x^t = 1 \) holds only when \( t = 0 \). Therefore, \( y_1 - y_2 = 0 \), which means \( y_1 = y_2 \). 
    This shows that \( f \) is injective because no two different inputs produce the same output.
\end{proof}

\section*{Question 2}
\begin{proof}
    \textit{The following proof follows the given proof outline:}
    \\
    \textbf{Step i:} For any $n \in \mathbb{N}$, $x - 1 \geq n(x^{\frac{1}{n}} - 1)$
    \\
    Recall Bernoulli's inequality, it states that for any real number $r \geq 1$ and $s \geq -1$:
    \[ (1 + s)^r \geq 1 + rs \]
    Let $s = x^{\frac{1}{n}} - 1$ (NOTE: $s > 0$ as $x > 1$) and $r = n$:
    \[ (1 + x^{\frac{1}{n}} - 1)^n \geq 1 + n(x^{\frac{1}{n}} - 1) \]
    \[ \Rightarrow x \geq 1 + n(x^{\frac{1}{n}} - 1) \Rightarrow x - 1 \geq n(x^{\frac{1}{n}} - 1) \]
    \textbf{Step ii:} If $t > 1$ and $n \in \mathbb{N}$ s.t. $n > \frac{x - 1}{t - 1}$, then $x^\frac{1}{n} < t$.
    \\
    From \textit{step i}, we get:
    \[ x^{\frac{1}{n}} - 1 \leq \frac{x - 1}{n} \]
    If $n > \frac{x - 1}{t - 1}$, then:
    \[ \frac{x - 1}{n} < t - 1 \]
    Therefore,
    \[ x^{\frac{1}{n}} - 1 < t - 1 \Longrightarrow x^{\frac{1}{n}} < t \]
    \textbf{Step iii:} If $y \in \mathbb{R}$ and $x^y < z$, then there exists $n \in \mathbb{N}$ s.t. $x^{y + \frac{1}{n}} < z$:
    \\
    Set $t = \frac{x^y}{z}$, note $\frac{x^y}{z} > 1$ by definition. Again using step (ii), choose $n \in \mathbb{N}$ s.t.:
    \[ n > \frac{x - 1}{t - 1} \]
    Then $x^{\frac{1}{n}} < t$, recall question $1(c)$ and multiply $x^y$ on both side, we can get:
    \[ x^y \cdot x^{\frac{1}{n}} = x^{y + \frac{1}{n}}  < x^y \cdot t = z \]
    \textbf{Step iv:} If $y \in \mathbb{R}$ and $x^y > z$, then there exists $n \in \mathbb{N}$ s.t. $x^{y - \frac{1}{n}} > z$:
    \\
    Set $t = \frac{z}{x^y}$, note $\frac{z}{x^y} > 1$ by definition. Again using step (ii), choose $n \in \mathbb{N}$ s.t.:
    \[ n > \frac{x - 1}{t - 1} \]
    Then $x^{\frac{1}{n}} < t$, so $ \frac{1}{x^{\frac{1}{n}}} = x^{-\frac{1}{n}} > \frac{1}{t}$. Recall question $1(c)$ and multiply $x^y$ on both side, we can get:
    \[ x^y \cdot x^{-\frac{1}{n}} = x^{y - \frac{1}{n}} > x^y \cdot \frac{1}{t} = z \]
    \textbf{Step v:} Define $A(z) = \{w \in \mathbb{R} | x^w < z \}$. Let $y = \sup A(z)$. then $x^y = z$
    \\
    \textit{Case 1: }
    Suppose $x^y < z$: by step (iii), there exists $n$ s.t. $x^{y + \frac{1}{n}} < z$. This indicates $y + \frac{1}{n} \in A(z)$. Noting that $x^{y + \frac{1}{n}} > x^{y}$, we can get:
    \[ z > x^{y + \frac{1}{n}} > x^{y} \]
    This contradicting the fact that $y = \sup A(z)$.
    \\
    \textit{Case 2: }
    Similarly, suppose $x^y > z$: by step (iv), there exists $n$ s.t. $x^{y - \frac{1}{n}} > z$. But for $y - \frac{1}{n}$, we can find a $w_0 \in A(z)$ s.t.
    \[ y - \frac{1}{n} < w < z \Rightarrow x^{y - \frac{1}{n}} < x^w < x^z \]
    This contradicting with the result that we have in step (iv)
    \\
    Consequently, neither $x^y < z$ not $x^y > z$ is possible; thus, $x^y = z$.
\end{proof}

\section*{Question 3}
By analyzed several terms of the given sequence, likely it is increasing and should converges to $2$.
\begin{proof}
    For the given sequence, we can define it recusively:
    \[ a_1 = \sqrt{2}, \ a_{n + 1} = \sqrt{2a_n}, \ for \ n \geq 1 \]
    First, we need to set up an induction to prove this sequence is increasing. I.e. We will prove $a_n < a_{n + 1}$ for all $n \geq 1$
    \\
    \textit{Base Case (n = 1):}
    \\
    \[ a_1 = \sqrt{2}, \ a_2 = \sqrt{2a_1} = \sqrt{2\sqrt{2}} \]
    Noticing that $a_1, a_2 > 0$, and $a_1^2 = 2, a_2^2 = 2\sqrt{2}$, so:
    \[ 2 < 2\sqrt{2} \Leftrightarrow a_1^2 < a_2^2 \Rightarrow a_1 < a_2 \]
    \\
    \textit{Inductive steps:}
    \\
    Assume $a_n < a_{n + 1}$ for some $n \geq 1$. We need to show $a_{n + 1} < a_{n + 2}$.
    \\
    Noticing, $a_{n + 1}$ and $a_{n + 2}$ can be deonte as:
    \[ a_{n + 1} = \sqrt{2a_n}, \ a_{n + 2} = \sqrt{2a_{n+1}} \]
    Then, since $a_n > 0, \ \forall n \in \mathbb{N}$:
    \[ \sqrt{2a_n} < \sqrt{2a_{n+1}} \Leftrightarrow 2a_n < 2a_{n+1} \Leftrightarrow a_n < a_{n+1} \]
    Hence, by our induction hypothesis, $a_{n + 1} < a_{n + 2}$. 
    \\
    Therefore by induction, sequence $\{a_n\}$ is increasing.
    \\
    Then, we need to show that $(a_n)$ is bounded above. (I.e., we will prove by induction that $a_n < 2$ for all $n \geq 1$)
    \\
    \textit{Base Case (n = 1):}
    \\
    \[ a_1 = \sqrt{2} < 2 \]
    \\
    \textit{Inductive Step:}
    \\
    Assume $a_n < 2$ for some $n \geq 1$, we need to show $a_{n + 1} < 2$:
    \\
    From our previous definition, $a_{n + 1} = \sqrt{2 a_n}$, we can get: 
    \[ a_{n + 1} = \sqrt{2a_n} < \sqrt{4} = 2 \]
    By induction again, $a_n < 2$ for all $n  1$.
    \\
    Since $\{a_n \}$ is increasing and bounded above by $2$, by MCT, $\{a_n\}$ converges to some limit $L$ s.t. $L \leq 2$
    \\
    Let $L = \lim_{n \rightarrow \infty} a_n$. Taking limit both sides of the recursive formula:
    \[ L = \lim_{n \rightarrow \infty} a_{n + 1} = \lim_{n \rightarrow \infty} \sqrt{2a_n} = \sqrt{2L} \]
    Then:
    \[ L = \sqrt{2L} \Longrightarrow L^2 = 2L \Longrightarrow L^2 - 2L = 0 \Longrightarrow L(L - 2) = 0 \]
    So $L = 0, 2$, and since $a_n > 0$, limit $L \geq \sqrt{2} > 0$. so $L = 2$.
    \\
    Lastly, the limit found for the given sequence is:
    \[ \lim_{n \rightarrow \infty} a_n = 2 \]    
\end{proof}

\end{document}
