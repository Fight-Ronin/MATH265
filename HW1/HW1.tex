\documentclass{article}
\usepackage{graphicx} % Required for inserting images
\usepackage[utf8]{inputenc}
\usepackage{amsmath}
\usepackage{graphicx}
\usepackage{tikz}
\usepackage{array}
\usepackage{amssymb}
\newcommand*{\twoheadrightarrowtail}{\mathrel{\rightarrowtail\kern-1.9ex\twoheadrightarrow}}
% Alternative which doesn't look as good using the normal size, but might work better with smaller sizes too:
%\newcommand*{\twoheadrightarrowtail}{\mathrel{\rlap{$\rightarrowtail$}\twoheadrightarrow}}
\usepackage{amssymb}
\usepackage{amsthm}
\usepackage{multirow}
\usepackage{dcolumn}
\newcolumntype{2}{D{.}{}{2.0}}

\title{MATH 240 HW1}
\author{Hanzhang Yin}
\date{Aug/26/2024}

\begin{document}

\maketitle

\section*{Question 1}
\begin{proof}
    Define a function $f$ as:
    \[ f: \mathbb{N} \rightarrow \{1,4,7,10,13,\cdots\}: n \rightarrowtail 3n-2 \]
    Now we need to show $f$ is \textit{one to one} and \textit{onto}.
    \begin{itemize}
        \item Prove $f$ is \textit{one to one}
        \begin{proof}
            To prove $f$ is \textit{one to one}, we must show that if $f(n_1) = f(n_2)$, then $n_1 = n_2$.
            \\
            Suppose $f(n_1) = f(n_2)$. Then:
            \[ 3n_1 - 2 = 3n_2 - 2 \]
            \[ \Rightarrow 3n_1 = 3n_2 \Rightarrow n_1 = n_2\]
            \\
            Since $n_1 = n_2$, $f$ is injective.
        \end{proof}
        \item Prove $f$ is \textit{onto}
        \begin{proof}
            To prove $f$ is \textit{onto}, we need to show for every $m \in \{1,4,7,10,13,\cdots\}$, there exists an $n \in \mathbb{N}$ s.t. $f(n) = m$.
            \\
            Take any $m \in \{1,4,7,10,13,\cdots\}$. We want to find $n$ such that $3n-2 = m$. Solving for $n$:
            \[ 3n-2 = m \Rightarrow 3n = m + 2 \Rightarrow n = \frac{m+2}{3} \]
            \\
            Since $m$ is of the form $3k-2$ for some integer $k$, and $n=k$ is an integer. Therefore, $n \in \mathbb{N}$, and $f$ is onto.
        \end{proof}
        Since $f$ is \textit{one to one} and \textit{onto}, it is a bijection. Hence, $f(n) = 3n - 2$ is a explicit bijection from the set $\mathbb{N}$ to $\{1,4,7,10,13,\cdots\}$.
    \end{itemize}
\end{proof}

\section*{Question 2}
\begin{proof}
    Define a function $f$ as:
    \[ f: (0,1) \rightarrow \mathbb{R}_{>0}: n \rightarrowtail \frac{n}{1-n} \]
    Now we need to show $f$ is \textit{one to one} and \textit{onto}.
    \begin{itemize}
        \item Prove $f$ is \textit{one to one}
        \begin{proof}
            To prove $f$ is \textit{one to one}, we must show that if $f(n_1) = f(n_2)$, then $n_1 = n_2$.
            \\
            Suppose $f(n_1) = f(n_2)$. Then:
            \[ \frac{n_1}{1 - n_1} = \frac{n_2}{1 - n_2} \]
            \[ \Rightarrow x_1(1-x_2) = x_2(1-x_1) \]
            \[ \Rightarrow x_1 - x_1x_2 = x_2 - x_2x_1 \]
            \[ \Rightarrow x_1 = x_2 \]
            \\
            Thus, $f$ is one to one.
        \end{proof}
        \item Prove $f$ is \textit{onto}
        \begin{proof}
            To prove $f$ is \textit{onto}, we need to show for every $m \in \mathbb{R}_{>0}$, there exists an $n \in (0,1)$ s.t. $f(n) = m$.
            \\
            Take any $m > 0$. We want to find $n$ such that $\frac{n}{1 - n} = m$. Solving for $n$:
            \[ n = m(1 - n) \Rightarrow n + nm = m \Rightarrow n(1 + m) = m \Rightarrow n = \frac{m}{1 + m} \]
            Note that $m > 0$, this implies $0 < \frac{m}{1 + m} < 1$, hence $n \in (0,1)$.
            \\
            Thus, for every $m > 0$, there is an $n \in (0,1)$ s.t. $f(n) = m$. (i.e. $f$ is onto)
        \end{proof}
        Since $f$ is \textit{one to one} and \textit{onto}, it is a bijection. Hence, $f(n) = \frac{n}{1-n}$ is a explicit bijection from the set $(0,1)$ to $\mathbb{R}_{>0}$.
    \end{itemize}
\end{proof}

\section*{Question 3}
\begin{proof}
    Modified the function $f$ from Q2 to $g$ as:
    \[ g: [0,1) \rightarrow \mathbb{R}_{>0} 
    : g(x) =
        \begin{cases}
            \frac{x}{1-x}, & \text{if } x \in (0, 1) \\
            a, & \text{if } x = 0
        \end{cases}
    \]
    Where $a \in \mathbb{R}_{+}$ and is not in the range of $f(x)$ for $x \in (0,1)$.
    \\
    Now we need to show $g$ is \textit{one to one} and \textit{onto}.
    \begin{itemize}
        \item Prove $g$ is \textit{one to one}
        \begin{proof}
            To prove $g$ is \textit{one to one}, we must show that if $g(x_1) = g(x_2)$, then $x_1 = x_2$.
            \\
            \begin{itemize}
                \item If $x_1, x_2 \in (0,1)$, then from Q2, $\frac{x_1}{1-x_1} = \frac{x_2}{1-x_2}$ implies $x_1 = x_2$
                \item If $x_1 = 0$ and $x_2 \in (0,1)$, then $g(0) = a, g(x_2) = \frac{x_2}{1-x_2}$, but $a \neq \frac{x_2}{1-x_2}, \forall x_2 \in (0,1)$.
                Thus, $g(x_1) \neq g(x_2)$ andit it is one to one.
            \end{itemize}
        \end{proof}
        \item Prove $g$ is \textit{onto}
        \begin{proof}
            To prove $g$ is \textit{onto}, we need to show for every $y \in \mathbb{R}_{>0}$, there exists an $x \in (0,1)$ s.t. $g(x) = y$.
            \\
            \begin{itemize}
                \item For $y \neq a$, there exists an $x \in (0,1)$ s.t. $\frac{x}{1-x} = y$.
                \item For $y = a$, we have $g(0) = a$.
            \end{itemize}
            Hence, $g(x)$ covers $\mathbb{R}_{+}$, which is onto.
        \end{proof}
        Since $g$ is \textit{one to one} and \textit{onto}, it is a bijection.
    \end{itemize}
\end{proof}

\section*{Question 4}
\begin{proof}

    \subsection*{(a) Show that $g \circ f$ is injective if both of $f$ and $g$ are injective.}

    Given that $f : A \rightarrow B$ and $g : B \rightarrow C$ are injective. 
    We can write $g \circ f$ as $g \circ f: A \rightarrow C$, defined as $(g \circ f)(x) = g(f(x))$.
    \\
    Assume: $(g \circ f)(x) = (g \circ f)(y), \ \text{for some $x,y \in A$} $, 
    \begin{enumerate}
        \item By the definition of composite function, we can get $g(f(x)) = g(f(y))$
        \item Given, $g$ is injective, $f(x) = f(y)$
        \item Also given, $f$ is injective, $x = y$
    \end{enumerate}
    \\
    Hence, $g \circ f$ is injective. 
    $\blacksquare$

    \subsection*{(b) Show that $g \circ f$ is surjective if both of $f$ and $g$ are surjective.}

    Given that $f : A \rightarrow B$ and $g : B \rightarrow C$ are surjective. 
    We can write $g \circ f$ as $g \circ f: A \rightarrow C$, defined as $(g \circ f)(x) = g(f(x))$.
    \\
    Let an arbitary value $z \in C$. We need to find some $x \in A$ such that $(g \circ f)(x) = z$
    \begin{enumerate}
        \item Given g is surjective, hence there exists a $y \in B$ such that $g(y) = z$
        \item Given f is surjective, hence there exists a $x \in B$ such that $f(x) = y$
        \item Hence, $(g \circ f)(x) = g(f(x)) = g(y) = z$
    \end{enumerate}
    \\
    Hence, $g \circ f$ is surjective. 
    $\blacksquare$
\end{proof}

\end{document}