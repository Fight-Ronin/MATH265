\documentclass{article}
\usepackage{graphicx} % Required for inserting images
\usepackage[utf8]{inputenc}
\usepackage{amsmath}
\usepackage{graphicx}
\usepackage{tikz}
\usepackage{array}
\usepackage{amssymb}
\newcommand*{\twoheadrightarrowtail}{\mathrel{\rightarrowtail\kern-1.9ex\twoheadrightarrow}}
% Alternative which doesn't look as good using the normal size, but might work better with smaller sizes too:
%\newcommand*{\twoheadrightarrowtail}{\mathrel{\rlap{$\rightarrowtail$}\twoheadrightarrow}}
\usepackage{amssymb}
\usepackage{amsthm}
\usepackage{multirow}
\usepackage{dcolumn}
\newcolumntype{2}{D{.}{}{2.0}}

\title{MATH 240 HW1}
\author{Hanzhang Yin}
\date{Aug/26/2024}

\begin{document}

\maketitle

\section*{Question 1}
\begin{proof}
    Define a function $f$ as:
    \[ f: \mathbb{N} \rightarrow \{1,4,7,10,13,\cdots\}: n \mapsto 3n-2 \]
    Now we need to show $f$ is \textit{one to one} and \textit{onto}.
    \begin{itemize}
        \item Prove $f$ is \textit{one to one}
        \begin{proof}
            To prove $f$ is \textit{one to one}, we must show that if $f(n_1) = f(n_2)$, then $n_1 = n_2$.
            \\
            Suppose $f(n_1) = f(n_2)$. Then:
            \[ 3n_1 - 2 = 3n_2 - 2 \]
            \[ \Rightarrow 3n_1 = 3n_2 \Rightarrow n_1 = n_2\]
            \\
            Since $n_1 = n_2$, $f$ is injective.
        \end{proof}
        \item Prove $f$ is \textit{onto}
        \begin{proof}
            To prove $f$ is \textit{onto}, we need to show for every $m \in \{1,4,7,10,13,\cdots\}$, there exists an $n \in \mathbb{N}$ s.t. $f(n) = m$.
            \\
            Take any $m \in \{1,4,7,10,13,\cdots\}$. We want to find $n$ such that $3n-2 = m$. Solving for $n$:
            \[ 3n-2 = m \Rightarrow 3n = m + 2 \Rightarrow n = \frac{m+2}{3} \]
            \\
            Since $m$ is of the form $3k-2$ for some integer $k$, and $n=k$ is an integer. Therefore, $n \in \mathbb{N}$, and $f$ is onto.
        \end{proof}
        Since $f$ is \textit{one to one} and \textit{onto}, it is a bijection. Hence, $f(n) = 3n - 2$ is a explicit bijection from the set $\mathbb{N}$ to $\{1,4,7,10,13,\cdots\}$.
    \end{itemize}
\end{proof}

\section*{Question 2}
\begin{proof}
    Define a function $f$ as:
    \[ f: (0,1) \rightarrow \mathbb{R}_{>0}: n \mapsto \frac{n}{1-n} \]
    Now we need to show $f$ is \textit{one to one} and \textit{onto}.
    \begin{itemize}
        \item Prove $f$ is \textit{one to one}
        \begin{proof}
            To prove $f$ is \textit{one to one}, we must show that if $f(n_1) = f(n_2)$, then $n_1 = n_2$.
            \\
            Suppose $f(n_1) = f(n_2)$. Then:
            \[ \frac{n_1}{1 - n_1} = \frac{n_2}{1 - n_2} \]
            \[ \Rightarrow n_1(1-n_2) = n_2(1-n_1) \]
            \[ \Rightarrow n_1 - n_1n_2 = n_2 - n_2n_1 \]
            \[ \Rightarrow n_1 = n_2 \]
            \\
            Thus, $f$ is one to one.
        \end{proof}
        \item Prove $f$ is \textit{onto}
        \begin{proof}
            To prove $f$ is \textit{onto}, we need to show for every $m \in \mathbb{R}_{>0}$, there exists an $n \in (0,1)$ s.t. $f(n) = m$.
            \\
            Take any $m > 0$. We want to find $n$ such that $\frac{n}{1 - n} = m$. Solving for $n$:
            \[ n = m(1 - n) \Rightarrow n + nm = m \Rightarrow n(1 + m) = m \Rightarrow n = \frac{m}{1 + m} \]
            Note that $m > 0$, this implies $0 < \frac{m}{1 + m} < 1$, hence $n \in (0,1)$.
            \\
            Thus, for every $m > 0$, there is an $n \in (0,1)$ s.t. $f(n) = m$. (i.e. $f$ is onto)
        \end{proof}
        Since $f$ is \textit{one to one} and \textit{onto}, it is a bijection. Hence, $f(n) = \frac{n}{1-n}$ is a explicit bijection from the set $(0,1)$ to $\mathbb{R}_{>0}$.
    \end{itemize}
\end{proof}

\section*{Question 3}
\begin{proof}
    Let $n_i \in (0,1)$ s.t. $f(n_i) = i$, modified the function $f$ from Q2 to $g$ as:
    \[ g: [0,1) \rightarrow \mathbb{R}_{>0} 
    : g(n) =
        \begin{cases}
            f(n), & \text{if } n \notin \{n_i\}_{i=1}^{\infty}, \ n \neq 0 \\
            i, & \text{if } n = n_i \ \text{for some i} \\
            1, & \text{if } n = 0
        \end{cases}
    \]
    Now we need to show $g$ is \textit{one to one} and \textit{onto}.
    \begin{itemize}
        \item Prove $g$ is \textit{one to one}
        \begin{proof}
            To prove $g$ is \textit{one to one}, we must show that if $g(x_1) = g(x_2)$, then $x_1 = x_2$.
            \\
            \textbf{Case 1:} \( x_1, x_2 \in (0, 1) \):
            \begin{itemize}
                \item If \( x_1, x_2 \notin \{n_i\}_{i=1}^\infty \), then \( g(x_1) = f(x_1) \) and \( g(x_2) = f(x_2) \). Since \( f \) is injective, \( f(x_1) = f(x_2) \implies x_1 = x_2 \)
                \item If \( x_1 = n_i \) for some \( i \) and \( x_2 = n_j \) for some \( j \), then \( g(x_1) = i \) and \( g(x_2) = j \). If \( g(x_1) = g(x_2) \), then \( i = j \), implying \( x_1 = n_i = n_j = x_2 \).
                \item If \( x_1 \notin \{n_i\}_{i=1}^\infty \) and \( x_2 = n_i \) for some \( i \) (or vice versa), then \( g(x_1) = f(x_1) \neq i = g(x_2) \). Thus, \( g(x_1) \neq g(x_2) \).
                Thus, $g(n_1) \neq g(n_2)$, showing it is one to one.
            \end{itemize}
            
            \textbf{Case 2:} \( x_1 = 0 \) and \( x_2 \in (0,1) \):
            \\
            Note $g(0) = 1$, if $x_2 \in (0,1)$, $g(x_2) \neq 1$ since $f(x) > 1, \forall x \in (0,1)$. Therefore, $g(0) \neq g(x_2)$.
        \end{proof}
        \item Prove $g$ is \textit{onto}
        \begin{proof}
            To prove $g$ is \textit{onto}, we need to show for every $y \in \mathbb{R}_{>0}$, there exists an $x \in [0,1)$ s.t. $g(x) = y$.
            \\
            \textbf{Case 1:} \( y = i \) for some \(i \in \mathbb{N}\):
            \\
            For each $i \in \mathbb{N}$, we have $g(n_i) = i$, ehere $n_i \in (0,1)$. Therefore, for every integer $i \in \mathbb{N}$, we can find some $n_i \in (0,1)$ s.t. $g(n_1) = i$.
            
            \textbf{Case 2:} \( y = 1 \):
            \\
            For $y = 1$, $g(0) = 1$. $y = 1$ is covered by $x = 0$.

            \textbf{Case 3:} \( y \in \mathbb{R}_{>0} \backslash \mathbb{N} \):
            \\
            From Q2, since $f$ is a bijection, for every $y \in \mathbb{R}_{>0} \backslash \{i | i \in \mathbb{N}\}$, there exists an $x \in (0,1) \backslash \{n_i\}_{i=1}^{\infty}$ s.t. $f(x) = y$.
            \\
            Hence, $g(x)$ covers $\mathbb{R}_{>0}$. (i.e. $g(x)$ is onto.)
        \end{proof}
        Since $g$ is \textit{one to one} and \textit{onto}, it is a bijection.
    \end{itemize}
\end{proof}

\section*{Question 4}

\subsection*{(a) Show that $g \circ f$ is injective if both of $f$ and $g$ are injective.}

\begin{proof}
    Given that $f : A \rightarrow B$ and $g : B \rightarrow C$ are injective. 
    We can write $g \circ f$ as $g \circ f: A \rightarrow C$, defined as $(g \circ f)(x) = g(f(x))$.
    \\
    Assume: $(g \circ f)(x) = (g \circ f)(y), \ \text{for some $x,y \in A$} $, 
    \begin{enumerate}
        \item By the definition of composite function, we can get $g(f(x)) = g(f(y))$
        \item Given, $g$ is injective, $f(x) = f(y)$
        \item Also given, $f$ is injective, $x = y$
    \end{enumerate}
    \\
    Hence, $g \circ f$ is injective. 
\end{proof}

\subsection*{(b) Show that $g \circ f$ is surjective if both of $f$ and $g$ are surjective.}

\begin{proof}
    Given that $f : A \rightarrow B$ and $g : B \rightarrow C$ are surjective. 
    We can write $g \circ f$ as $g \circ f: A \rightarrow C$, defined as $(g \circ f)(x) = g(f(x))$.
    \\
    Let an arbitary value $z \in C$. We need to find some $x \in A$ such that $(g \circ f)(x) = z$
    \begin{enumerate}
        \item Given g is surjective, hence there exists a $y \in B$ such that $g(y) = z$
        \item Given f is surjective, hence there exists a $x \in B$ such that $f(x) = y$
        \item Hence, $(g \circ f)(x) = g(f(x)) = g(y) = z$
    \end{enumerate}
    \\
    Hence, $g \circ f$ is surjective.
\end{proof}

\end{document}