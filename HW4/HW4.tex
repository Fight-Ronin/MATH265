\documentclass{article}
\usepackage{graphicx} % Required for inserting images
\usepackage[utf8]{inputenc}
\usepackage{amsmath}
\usepackage{graphicx}
\usepackage{tikz}
\usepackage{array}
\usepackage{amssymb}
\newcommand*{\twoheadrightarrowtail}{\mathrel{\rightarrowtail\kern-1.9ex\twoheadrightarrow}}
% Alternative which doesn't look as good using the normal size, but might work better with smaller sizes too:
%\newcommand*{\twoheadrightarrowtail}{\mathrel{\rlap{$\rightarrowtail$}\twoheadrightarrow}}
\usepackage{amssymb}
\usepackage{amsthm}
\usepackage{multirow}
\usepackage{verbatim}
\usepackage{dcolumn}
\newcolumntype{2}{D{.}{}{2.0}}

\title{MATH 265 HW4}
\author{Hanzhang Yin}
\date{Sep/18/2024}

\begin{document}

\maketitle

\section*{Question 1}
\begin{proof}
    Given \( w \) is a lower bound of \( S \), by definition:
    \[ w \leq s, \quad \forall s \in S. \]
    Since \( w \in S \), there exists a \(s \in S\) s.t. \( s = w \). Then,
    \[ w \leq s = w \implies w = s. \]
    Suppose there exists another lower bound \( v \in \mathbb{R} \) of \( S \) such that \( v > w \).
    However, since \( w \in S \), it must satisfy:
    \[ v \leq w. \]
    we derived a contradiction.
    Therefore, no such \( v \) exists, and \( w \) must be the g.l.b. of $S$
\end{proof}

\section*{Question 2}
\begin{proof}
    First, we want to show $\inf(A) = m$ is an upper bound of $-A$.
    \\
    Let $\inf(A) = m$, by definition of infimum, $m \leq a$ for all $a \in A$.
    \[ \Rightarrow -m \geq -a, \forall a \in A \]
    By our definition of set $-A$, this means $-\inf(A)$ is an upper bound of $-A$.
    \\
    Following that, now we can show $-\inf(A)$ is the l.u.b of $-A$:
    \\
    Assume there exists an upper bound $u \in -A$ s.t. $u < -\inf(A) = -m$. Then:
    \[ u < -m \rightarrow -u > m \]
    Since $\inf(A) = m$, for $\epsilon = -u - m > 0$, there exists $a_{\epsilon} \in A$ s.t. 
    \[ a_{\epsilon} < m + \epsilon = m + (-u - m) = -u \Rightarrow -a_{\epsilon} > u \]
    Hence, $u$ is NOT an upper bound of $-A$, and $-m \leq u$ always. This shows that $-\inf(A) = \sup(-A)$.
\end{proof}

\section*{Question 3}

\subsection*{(a)}
\begin{proof}
    Since \( A \cap B \subseteq A \) and \( A \cap B \subseteq B \), any upper bound of \( A \) or \( B \) is also an upper bound of \( A \cap B \).
    \\
    Let \( \sup A = u \). By definition, \( u \) is the least upper bound of \( A \), hence \( u \) is an upper bound for \( A \cap B \).
    Similarly, let \( \sup B = w \). Then \( w \) is an upper bound for \( A \cap B \).
    \\
    Since \( \sup(A \cap B) \) is the least upper bound of \( A \cap B \), it must satisfy:
    \[ \sup(A \cap B) \leq u = \sup A \quad \text{and} \quad \sup(A \cap B) \leq w = \sup B. \]
    Therefore,
    \[ \sup A, \sup B \geq \sup(A \cap B). \]
\end{proof}

\subsection*{(b)}

\end{document}
